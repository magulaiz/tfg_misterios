\cleardoublepage
\section*{Resumen}

El presente trabajo ofrece un análisis comparativo entre dos de los cultos mistéricos más significativos de la religión griega antigua: los misterios eleusinos y los misterios de Samotracia. A través de una investigación de carácter cualitativo, se estudian aspectos como el ámbito geográfico, la funcionalidad del culto, las divinidades asociadas, el espacio sagrado, el proceso de iniciación, los participantes, los mitos fundacionales, las etapas festivas y la duración del culto. El estudio evidencia que, si bien ambos cultos compartieron ciertas características estructurales y sociales, respondieron a concepciones diferentes sobre la relación del ser humano con la divinidad, la naturaleza y el más allá.


\section*{Abstract}

The present work offers a comparative analysis between the most significant mystery cults of ancient Greek religion: the Eleusinian mysteries and the Samotracian mysteries. Through a qualitative research, aspects such as the geographical scope, the functional nature of the cult, the associated divinitats, the sacred space, the initiation process, the participants, the founding mites, the festive stages and the duration of the cult are studied. corresponding The study shows that, if all two cultures share certain structural and social characteristics, they respond to different conceptions about the relationship between human beings and divinity, nature and beyond.
