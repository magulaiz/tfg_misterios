\chapter{Introducción}

El presente trabajo trata sobre los cultos mistéricos que ocuparon la antigua religión griega. En concreto, se centra en un estudio comparativo sobre los misterios de Eleusis, dedicados a Deméter y Perséfone y los misterios de Samotracia, dedicado a los cabiros. Ambos cultos surgieron en un mismo contexto histórico y cultural: el mundo greco-romano; esto quiere decir que tuvieron que lidiar con los diferentes eventos sucedidos, ya fueran crisis sociales, conflictos políticos o cambios culturales. Ambas, también, compartieron puntos de vista en cuanto a cómo debe vivirse la vida para alcanzar la salvación o la verdad.

Antes de entrar en materia me gustaría realizar, a modo de resumen, lo que fue la religión griega, ya que considero pertinente enmarcar los cultos mistéricos en una totalidad.

La Grecia antigua estuvo marcada por una heterogeneidad religiosa, aunque los pueblos helénicos gozaban de unión gracias a compartir etnicidad, lengua, cultura y religión. Esta heterogeneidad, estuvo definida, según Alfonso Reyes \footcites{reyesReligionGriega2018}, por diferentes puntos que expondremos a continuación.

En primer lugar, la diversidad étnica que existía en el territorio no se produjo unicamente a causa de los conflictos armados, sino debido, principalmente, a una progresiva penetración de poblaciones de otras culturas, como la protagonizada por los aqueos y los dorios en el siglo XII a. C. , que ya estaban familiarizados con la cultura egea. Esto produjo una combinación de culturas y tradiciones en el mundo egeo, generando así una nueva identidad. Es así como se comienza a dibujar una nueva religión griega en la que reinan los Olímpicos de apariencia humana, a partir de los cuales se genera toda una mitología.

Cada polis o región podía tener su propia tradición, con variantes leves o significativas. Podemos destacar como 'marcador' de esta disparidad en la religión, la amplia nomenclatura religiosa para las deidades; y con esto nos referimos a los diferentes epítetos o descripciones que podía recibir una divinidad, dependiendo, sobre todo, del ámbito geográfico donde se instaurara su uso. Aun así, dentro de esta pluralidad, cada epíteto y denominación se refiere a una única deidad conocida en todo el territorio griego. 

Otro punto relevante es la ausencia de un organismo jerarquizado y regulador del culto, donde un cuerpo sacerdotal especializado actúe como intermediario con lo divino, a modo de iglesia ecuménica. Al contrario, el creyente individualiza su vivencia religiosa adaptándola a su manera de concebir y entender sus creencias\footcite[47]{reyesReligionGriega2018} (vivencia personal de la creencia y práctica religiosa).  

En cuanto a la literatura, resalta la libertad de los autores griegos a la hora de producir su obra y describir la genealogía y los  atributos de las divinidades o el relato de los mitos. Esto lo podemos observar, por ejemplo, en ciertas contradicciones entre los poemas homéricos y la  \textit{Teogonía} de Hesíodo. 

La religión se practicaba principalmente en torno del mito y el rito. Este último se fijaba según el lugar y periodo, siguiendo un programa de acciones sagradas. El rito era el medio para crear y asegurar la solidaridad del grupo de creyentes y además le daba un sentido espiritual al mito. A mismo tiempo, el mito motivaba a partir de su relato, la necesidad del rito.

Los misterios (\textit{mystéria}), por su parte, tienen su origen en la cultura cretense, la cual realizó un intercambio cultural con Egipto y Asia Anterior, donde se veneraba a la Diosa Nutriz y a su acompañante denominado como paredro, personaje que se convertiría en Zeus posteriormente. Este culto se llevaba a cabo mediante ritos agrarios y sacramentos públicos (\textit{corum populo}). El establecimiento de los misterios fue una particularidad dentro de toda la generalidad de la religión griega, que como hemos mencionado, fue un conglomerado diverso.

Meyer\footcite[4]{w.meyerAncientMysteriesSource1986}, identifica el afloramiento de los misterios durante la etapa helenística, a raíz de las conquistas llevadas a cabo por Alejandro Magno. Los valores de la antigua polis griega se transformaron, y con ellos la concepción sobre las deidades olímpicas. El mundo helenístico era más cosmopolita y las divinidades provenían también de India, Asia, África y Anatolia. Aunque el panteón olímpico continuaba teniendo su poder religioso y cultural, comienzan a surgir nuevas interpretaciones filosóficas que buscaban explicar los textos mitológicos desde perspectivas racionalistas, cuestionando la naturaleza atribuida hasta entonces a los dioses. Los misterios arraigaron en este contexto y resultaron muy atractivos para quienes anhelaban vivencias religiosas diferentes. 

A grandes rasgos, un misterio se concibe como tal por las siguientes cualidades: pertenece a los iniciados (\textit{mýstes}) y debe guardarse en secreto; toda persona interesada en iniciarse es aceptada sin tener en cuenta su clase social, sexo, edad o lugar de origen; y además debían acompañarse de ritos exteriores y de algunas nociones éticas en cuanto a la vida y al 'más allá'\footcite[113]{reyesReligionGriega2018}. En contraposición a los cultos oficiales, en la que las personas mostraban una lealtad pública a los dioses, en los misterios destacaba la privacidad del culto en comunidades cerradas. Las personas que deciden individualmente ser iniciadas forman una colectividad unida entre sí al compartir una misma búsqueda de la salvación personal. Los iniciados en los misterios no debían divulgar el secreto revelado en la ceremonia, el cual podía depender de la representacióǹ formal y simbólica del misterio \footcite[5]{w.meyerAncientMysteriesSource1986}.

\section{Justificación del tema}

El presente trabajo se propone analizar y comparar dos de los cultos mistéricos más relevantes del mundo griego antiguo: los misterios de Eleusis y los de Samotracia. El interés por este tema surge de la importancia que estos ritos tuvieron en la religión griega, tanto por su duración como por su alcance social y geográfico. Ambos cultos ofrecían a sus participantes una experiencia religiosa distinta del culto público, centrada en la iniciación personal, la participación en rituales secretos y la promesa de algún tipo de beneficio espiritual, ya fuera protección, salvación o una existencia mejor tras la muerte.

La elección de estos dos casos concretos responde a que representan dos modelos distintos dentro del fenómeno mistérico. Eleusis está claramente vinculado a un mito definido y a un calendario ritual fijado, mientras que Samotracia presenta una estructura más flexible y unas divinidades menos identificables. A pesar de estas diferencias, ambos comparten elementos formales similares, como la existencia de un santuario, un proceso de iniciación y una comunidad de fieles que se reconocía como transformada por la experiencia ritual.

Analizar ambos cultos de manera comparativa permite no solo observar coincidencias y contrastes, sino también reflexionar sobre cómo se configuraba la experiencia religiosa en el ámbito mistérico y qué necesidades cubría dentro de la sociedad griega. Asimismo, este estudio ayuda a comprender mejor cómo se articulaban las creencias religiosas con otros aspectos de la vida antigua, como la política, el comercio o la identidad colectiva.


\section{Estado de la cuestión}

El estudio de los misterios eleusinos cuenta con una vasta bibliografía. Entre las fuentes primarias destaca el Himno Homérico a Deméter, clave para entender el mito fundacional del culto, así como los testimonios de Heródoto y autores cristianos como Clemente de Alejandría, quien a pesar de su rechazo ideológico, preservó valiosa información sobre los ritos. En cuanto a los estudios modernos, resaltan las aportaciones de Walter Burkert, quien en ``Ancient Mystery Cults'' establece una comparativa general entre los cultos mistéricos del mundo grecorromano; Karl Kerényi, que ofrece una visión simbólica y psicológica de Eleusis; y George Mylonas, quien documenta arqueológicamente el santuario eleusino.

Por su parte, los misterios de Samotracia han sido menos abordados, en parte por la naturaleza fragmentaria de sus fuentes. Sin embargo, destacan los estudios de Susan Guettel Cole en ``Theoi Megaloi'', que analiza la configuración del santuario y su función ritual, y los aportes de Hugh Bowden, quien estudia la relación entre mito, identidad y secreto en este culto. Recientemente, Carlos Moral ha propuesto un enfoque centrado en la influencia del culto samotracio sobre la vida marítima del mundo griego. Este trabajo pretende integrar estas aproximaciones en un análisis comparativo coherente que pone en diálogo ambos sistemas rituales desde una perspectiva histórico-religiosa.

\section{Metodología}

En este trabajo se desarrolla un estudio comparativo entre los misterios eleusinos y los misterios de Samotracia. Para llevarlo a cabo, se ha realizado  una toma de datos cualitativos para describir ambos cultos. Se hace una descripción -detallada por apartados- sobre cada culto, para luego proceder con el análisis comparativo. En este apartado, se tomarán en cuenta las coincidencias y diferencias, finalizando con una reflexión histórico-religiosa sobre los misterios como fenómeno social.

En cuanto a la cronología manejada en el trabajo, queda delimitada a partir del siglo VI a. C, cuando Atenas asume la jurisdicción de Eleusis, y Homero y Hesíodo comienzan a dibujar los dioses olímpicos. Podemos calificar este trabajo como ácrono, ya que la información y descripciones pertenecen a un gran intervalo de tiempo. Sí podemos determinar que la información obtenida sobre los misterios pertenece, sobre todo, al periodo de dominación ateniense. 

Por lo que respecta a las fuentes primarias, nos apoyamos inevitablemente en autores clásicos. Destacar entre ellos, el Himno homérico a Deméter, atribuido a Homero -sin entrar en su autoría real-; también la obra de Heródoto, gran historiador clásico. Resaltar la importancia de las fuentes primarias, ya que son testigo directo y una gran herramienta para desgranar la religión griega. Añadir que a lo largo de este estudio recurriremos a citas textuales de textos clásicos, por el hecho de que facilitan, a nuestro parecer, el seguimiento de las explicaciones. 

Autores contemporáneos son los más abundantes. Encontramos a Walter Burkert, de quien partió este trabajo; Kerenyi en el ámbito arqueológico y de significación de los misterios eleusinos y Alfonso Reyes, con su obra Religión Griega. Estudio importante es también el de Susan Guettel, dedicado a la arqueología y desarrollo del culto en Samotracia, igual que Hugh Bowden y su libro \textit{Mystery Cults in the Ancient World}. 

Trabajos actuales abundan, sobre todo artículos y revisiones, como el de Carlos Moral, el trabajo de Evans sobre Eleusis o el estudio de Stewart sobre la Niké de Samotracia.


