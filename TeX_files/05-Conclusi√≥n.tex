\chapter{Conclusión}
No se puede negar el hecho de que para el estudio de un misterio en específico se realiza una comparación, intencional o no, con otros misterios, no para observar las coincidencias o diferencias, sino para poder desentrañar los diferentes aspectos que conformaban el culto. 

punto de coincidencia: todas surgen en el mundo greco-romano, se enfrentaron a los mismos retos, propusieron similares formas de salvación y compartieron visiones similares sobre la luz y el camino de la vida\footcite
{w.meyerAncientMysteriesSource1986}.  

Las prácticas mistéricas tenían mucha relevancia en muchas facetas de la sociedad griega, llegando a existir una gran variedad de ellas perfectamente compatibles entre sí\footcite[79-80]{moralgarciaInfluenciaMisteriosSamotracia2020}. El análisis comparativo entre los misterios de Eleusis y los de Samotracia permite entender no solo las semejanzas estructurales entre ambos cultos, sino también sus profundas diferencias simbólicas y funcionales. Mientras Eleusis representa una religión de la tierra, del ciclo agrario y del renacimiento vital, Samotracia se vincula a la protección en el mar y a la redención individual. Ambos ofrecían una experiencia religiosa transformadora, accesible y profunda, que ayudaba al individuo a situarse en el mundo y a trascender su propia finitud. En este sentido, los misterios reflejan la complejidad de la religiosidad griega y su capacidad para integrar lo colectivo y lo personal, lo público y lo secreto, lo terrenal y lo trascendental.

Sin embargo, uno de los aspectos más desafiantes para los estudiosos modernos es comprender de manera auténtica cómo vivían y sentían los griegos estas iniciaciones. A pesar de los abundantes testimonios arqueológicos, literarios y epigráficos, el núcleo emocional y psicológico de la experiencia mistérica permanece en gran medida inaccesible. El secreto que rodeaba estos cultos no solo protegía los rituales de miradas profanas, sino que también construía una vivencia personal e intransferible, difícilmente traducible a categorías racionales. Lo sagrado, en estos contextos, se experimentaba; así dice Bowden \textit{"la experiencia misma de ser iniciado era lo que fundamental en estos cultos"} \footcite[83]{bowdenMysteryCultsAncient2023}. Y es precisamente esa dimensión interior, íntima, la que escapa a los registros históricos y deja al investigador contemporáneo ante un misterio no solo religioso, sino también epistemológico. Por tanto, cualquier aproximación al fenómeno mistérico debe reconocer, con humildad, los límites del conocimiento y abrirse a una comprensión más simbólica, fragmentaria y sugestiva de la espiritualidad griega antigua. 

